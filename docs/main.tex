% =========================================================
%  Respostas - Lista de Exercícios 1
%  Estrutura com duas partes:
%   (I) Respostas manuscritas (importadas de PDF/fotos)
%   (II) Respostas em código (Python)
% =========================================================
\documentclass[12pt,openany]{book}

% -------------------- Idioma e fontes --------------------
\usepackage[T1]{fontenc}
\usepackage[utf8]{inputenc}
\usepackage[brazil]{babel}

% -------------------- Layout e utilidades ----------------
\usepackage[a4paper,margin=2.5cm]{geometry}
\usepackage{graphicx}   % \includegraphics
\usepackage{pdfpages}   % \includepdf para páginas de um PDF
\usepackage{float}      % [H]
\usepackage{hyperref}
\hypersetup{
  colorlinks=true,
  linkcolor=blue!60!black,
  urlcolor=blue!60!black,
  citecolor=blue!60!black
}
\usepackage{xparse}     % comandos com argumentos opcionais
\usepackage{enumitem}   % listas compactas

% -------------------- Código (Python) --------------------
\usepackage{xcolor}
\usepackage{listings}

% Paleta mínima
\definecolor{pybg}{RGB}{248,248,248}
\definecolor{pyframe}{RGB}{220,220,220}
\definecolor{pyblue}{RGB}{32,64,154}
\definecolor{pygreen}{RGB}{34,139,34}
\definecolor{pystring}{RGB}{163,21,21}
\definecolor{pycomment}{RGB}{120,120,120}

\lstdefinestyle{pythonstyle}{
  language=Python,
  basicstyle=\ttfamily\small,
  keywordstyle=\color{pyblue}\bfseries,
  stringstyle=\color{pystring},
  commentstyle=\color{pycomment}\itshape,
  showstringspaces=false,
  columns=fullflexible,
  backgroundcolor=\color{pybg},
  frame=single,
  rulecolor=\color{pyframe},
  frameround=ffff,
  breaklines=true,
  tabsize=2,
  numbersep=8pt,
  numbers=left,
  numberstyle=\tiny\color{pycomment}
}

\lstdefinestyle{outputstyle}{
  basicstyle=\ttfamily\small,
  backgroundcolor=\color{pybg},
  frame=single,
  rulecolor=\color{pyframe},
  frameround=ffff,
  breaklines=true,
  numbers=none
}

% (Opcional) inclusão direta de arquivos .py:
% \lstinputlisting[style=pythonstyle]{caminho/arquivo.py}

% -------------------- Macros para inserir manuscritos ----
% Insere TODAS as páginas de um PDF escaneado:
\NewDocumentCommand{\HandwrittenPDFAll}{m}{%
  \includepdf[pages=-,pagecommand={\thispagestyle{plain}}]{#1}%
}

% Insere um intervalo/padrão de páginas de um PDF (ex.: 1-3,5):
\NewDocumentCommand{\HandwrittenPDFPages}{m m}{%
  \includepdf[pages={#2},pagecommand={\thispagestyle{plain}}]{#1}%
}

% Insere uma foto (JPG/PNG) de resposta manuscrita com legenda opcional:
\NewDocumentCommand{\HandwrittenImage}{m O{} }{%
  \begin{figure}[H]
    \centering
    \includegraphics[width=\linewidth]{#1}
    \IfNoValueTF{#2}{}{%
      \caption{#2}
    }
  \end{figure}
}

% Macro auxiliar
\newcommand{\CodeOutput}[2][]{%
  \lstinputlisting[style=outputstyle,caption={#1},label={#2}]{#2}
}

% -------------------- Metadados --------------------------
\title{Respostas \\ Lista de Exercícios 1}
\author{Reinan Gabriel dos Santos Souza}
\date{\today}

% =========================================================
\begin{document}
\maketitle
\tableofcontents

% =========================================================
\part{Respostas manuscritas}
\HandwrittenPDFAll{respostas_manuscritas.pdf}

% =========================================================
\part{Respostas em código}

\chapter{Códigos das questões}

Todos os códigos das questões apresentados neste documento fazem parte do meu repositório público disponível no GitHub:

\begin{center}
\href{https://github.com/ReinanHS/paa-ufs-lista01}{\texttt{github.com/ReinanHS/paa-ufs-lista01}}
\end{center}

O presente PDF é apenas um anexo para facilitar a leitura e organização. Entretanto, o local de referência principal é o repositório, onde é possível:

\begin{itemize}
  \item Visualizar diretamente os arquivos fonte em Python de cada questão.
  \item Clonar o repositório e executar os códigos localmente em sua própria máquina.
  \item Acompanhar a execução dos algoritmos por meio do pipeline de \textbf{CI/CD} configurado no GitHub Actions.
\end{itemize}

\bigskip
\noindent
A seguir, são listados os códigos de cada questão, acompanhados de links diretos para os arquivos correspondentes no GitHub:

\begin{itemize}[leftmargin=1.5cm]
  \item Q1a: \href{https://github.com/ReinanHS/paa-ufs-lista01/blob/main/src/q1/Q1a_insert_head.py}{\texttt{src/q1/Q1a\_insert\_head.py}}
  \item Q1b: \href{https://github.com/ReinanHS/paa-ufs-lista01/blob/main/src/q1/Q1b_insert_tail_no_tail.py}{\texttt{src/q1/Q1b\_insert\_tail\_no\_tail.py}}
  \item Q1c: \href{https://github.com/ReinanHS/paa-ufs-lista01/blob/main/src/q1/Q1c_second_smallest.py}{\texttt{src/q1/Q1c\_second\_smallest.py}}
  \item Q1d: \href{https://github.com/ReinanHS/paa-ufs-lista01/blob/main/src/q1/Q1d_sum_square_matrices.py}{\texttt{src/q1/Q1d\_sum\_square\_matrices.py}}
  \item Q1e: \href{https://github.com/ReinanHS/paa-ufs-lista01/blob/main/src/q1/Q1e_count_occurrences.py}{\texttt{src/q1/Q1e\_count\_occurrences.py}}
  \item Q2a: \href{https://github.com/ReinanHS/paa-ufs-lista01/blob/main/src/q2/Q2a_interval_sorted.py}{\texttt{src/q2/Q2a\_interval\_sorted.py}}
  \item Q2b: \href{https://github.com/ReinanHS/paa-ufs-lista01/blob/main/src/q2/Q2b_interval_unsorted.py}{\texttt{src/q2/Q2b\_interval\_unsorted.py}}
  \item Q6a: \href{https://github.com/ReinanHS/paa-ufs-lista01/blob/main/src/q3/Q6a_bfs_adj_matrix.py}{\texttt{src/q3/Q6a\_bfs\_adj\_matrix.py}}
  \item Q6b: \href{https://github.com/ReinanHS/paa-ufs-lista01/blob/main/src/q3/Q6b_dfs_adj_matrix_recursive.py}{\texttt{src/q3/Q6b\_dfs\_adj\_matrix\_recursive.py}}
  \item Q6c: \href{https://github.com/ReinanHS/paa-ufs-lista01/blob/main/src/q3/Q6c_dfs_adj_matrix_iterative.py}{\texttt{src/q3/Q6c\_dfs\_adj\_matrix\_iterative.py}}
\end{itemize}

\section{Questão 1}
\subsection*{Q1a) Inserção na cabeça de uma lista simplesmente ligada}
\lstinputlisting[style=pythonstyle, caption={Q1a: Inserção na cabeça}, label={lst:q1a}]{../src/q1/Q1a_insert_head.py}

\noindent\textbf{Output esperado:}
\begin{lstlisting}[style=outputstyle]
python src/q1/Q1a_insert_head.py
Lista: (vazia)
Lista: 1 -> 2 -> 3
\end{lstlisting}

\newpage

\subsection*{Q1b) Inserção no final (sem ponteiro de cauda)}
\lstinputlisting[style=pythonstyle, caption={Q1b: Inserção no final sem cauda}, label={lst:q1b}]{../src/q1/Q1b_insert_tail_no_tail.py}

\noindent\textbf{Output esperado:}
\begin{lstlisting}[style=outputstyle]
python src/q1/Q1b_insert_tail_no_tail.py
Lista: 10 -> 20 -> 30
\end{lstlisting}

\newpage

\subsection*{Q1c) Segundo menor elemento de um vetor}
\lstinputlisting[style=pythonstyle, caption={Q1c: Segundo menor elemento}, label={lst:q1c}]{../src/q1/Q1c_second_smallest.py}

\noindent\textbf{Output esperado:}
\begin{lstlisting}[style=outputstyle]
python src/q1/Q1c_second_smallest.py
Segundo menor de [80, 60, 90, 10, 40, 20] = 20
Segundo menor de [70, 80, 40, 30, 50] = 40
Segundo menor de [100, 100, 100] = None
\end{lstlisting}

\newpage

\subsection*{Q1d) Soma de matrizes quadradas}
\lstinputlisting[style=pythonstyle, caption={Q1d: Soma de matrizes n$\times$n}, label={lst:q1d}]{../src/q1/Q1d_sum_square_matrices.py}

\noindent\textbf{Output esperado:}
\begin{lstlisting}[style=outputstyle]
python src/q1/Q1d_sum_square_matrices.py

Matriz A:
  [10, 20]
  [30, 40]

Matriz B:
  [50, 60]
  [70, 80]

A + B:
  [60, 80]
  [100, 120]
\end{lstlisting}

\newpage

\subsection*{Q1e) Contagem de ocorrências em vetor desordenado}
\lstinputlisting[style=pythonstyle, caption={Q1e: Contagem de ocorrências}, label={lst:q1e}]{../src/q1/Q1e_count_occurrences.py}

\noindent\textbf{Output esperado:}
\begin{lstlisting}[style=outputstyle]
python src/q1/Q1e_count_occurrences.py
Ocorrências de 3 em [3, 8, 6, 3, 5, 4, 3] = 3
Ocorrências de 'a' em ['b', 'c', 'a', 'c', 'd', 'a'] = 2
\end{lstlisting}

\newpage

\section{Questão 2}
\subsection*{Q2a) Intervalo (max - min) em vetor ordenado}
\lstinputlisting[style=pythonstyle, caption={Q2a: Intervalo em vetor ordenado}, label={lst:q2a}]{../src/q2/Q2a_interval_sorted.py}

\noindent\textbf{Output esperado:}
\begin{lstlisting}[style=outputstyle]
python src/q2/Q2a_interval_sorted.py
Intervalo de [1, 4, 9] = 8
Intervalo de [-10, -5, 0, 5] = 15
\end{lstlisting}

\newpage

\subsection*{Q2b) Intervalo (max - min) em vetor desordenado}
\lstinputlisting[style=pythonstyle, caption={Q2b: Intervalo em vetor desordenado}, label={lst:q2b}]{../src/q2/Q2b_interval_unsorted.py}

\noindent\textbf{Output esperado:}
\begin{lstlisting}[style=outputstyle]
python src/q2/Q2b_interval_unsorted.py
Intervalo de [4, 9, 1] = 8
Intervalo de [-5, 10, 0, 5, -10] = 20
\end{lstlisting}

\newpage

\section{Questão 6}
\subsection*{Q6a) BFS com matriz de adjacência}
\lstinputlisting[style=pythonstyle, caption={Q6a: BFS matriz adjacência}, label={lst:q6a}]{../src/q3/Q6a_bfs_adj_matrix.py}

\noindent\textbf{Output esperado:}
\begin{lstlisting}[style=outputstyle]
python src/q3/Q6a_bfs_adj_matrix.py
BFS a partir de 0: [0, 1, 2]
BFS a partir de 1: [1, 0, 2]
\end{lstlisting}

\newpage

\subsection*{Q6b) DFS iterativa com matriz de adjacência}
\lstinputlisting[style=pythonstyle, caption={Q6c: DFS iterativa matriz adjacência}, label={lst:q6c}]{../src/q3/Q6c_dfs_adj_matrix_iterative.py}

\noindent\textbf{Output esperado:}
\begin{lstlisting}[style=outputstyle]
python src/q3/Q6c_dfs_adj_matrix_iterative.py
DFS (iter) a partir de 0: [0, 1, 2, 3]
DFS (iter) a partir de 2: [2, 0, 1, 3]
\end{lstlisting}

\end{document}